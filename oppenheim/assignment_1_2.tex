\documentclass[journal,12pt,twocolumn]{IEEEtran}
%
\usepackage{setspace}
\usepackage{gensymb}
\usepackage{xcolor}
\usepackage{caption}
%\usepackage{subcaption}
%\doublespacing
\singlespacing

%\usepackage{graphicx}
%\usepackage{amssymb}
%\usepackage{relsize}
\usepackage[cmex10]{amsmath}
\usepackage{mathtools}
%\usepackage{amsthm}
%\interdisplaylinepenalty=2500
%\savesymbol{iint}
%\usepackage{txfonts}
%\restoresymbol{TXF}{iint}
%\usepackage{wasysym}
\usepackage{hyperref}
\usepackage{amsthm}
\usepackage{mathrsfs}
\usepackage{txfonts}
\usepackage{stfloats}
\usepackage{cite}
\usepackage{cases}
\usepackage{subfig}
%\usepackage{xtab}
\usepackage{longtable}
\usepackage{multirow}
%\usepackage{algorithm}
%\usepackage{algpseudocode}
%\usepackage{enumerate}
\usepackage{enumitem}
\usepackage{mathtools}
%\usepackage{iithtlc}
%\usepackage[framemethod=tikz]{mdframed}
\usepackage{listings}
\let\vec\mathbf


%\usepackage{stmaryrd}


%\usepackage{wasysym}
%\newcounter{MYtempeqncnt}
\DeclareMathOperator*{\Res}{Res}
%\renewcommand{\baselinestretch}{2}
\renewcommand\thesection{\arabic{section}}
\renewcommand\thesubsection{\thesection.\arabic{subsection}}
\renewcommand\thesubsubsection{\thesubsection.\arabic{subsubsection}}

\renewcommand\thesectiondis{\arabic{section}}
\renewcommand\thesubsectiondis{\thesectiondis.\arabic{subsection}}
\renewcommand\thesubsubsectiondis{\thesubsectiondis.\arabic{subsubsection}}

%\renewcommand{\labelenumi}{\textbf{\theenumi}}
%\renewcommand{\theenumi}{P.\arabic{enumi}}

% correct bad hyphenation here
\hyphenation{op-tical net-works semi-conduc-tor}

\lstset{
	language=Python,
	frame=single, 
	breaklines=true,
	columns=fullflexible
}



\begin{document}
	%
	
	\theoremstyle{definition}
	\newtheorem{theorem}{Theorem}[section]
	\newtheorem{problem}{Problem}
	\newtheorem{proposition}{Proposition}[section]
	\newtheorem{lemma}{Lemma}[section]
	\newtheorem{corollary}[theorem]{Corollary}
	\newtheorem{example}{Example}[section]
	\newtheorem{definition}{Definition}[section]
	%\newtheorem{algorithm}{Algorithm}[section]
	%\newtheorem{cor}{Corollary}
	\newcommand{\BEQA}{\begin{eqnarray}}
		\newcommand{\EEQA}{\end{eqnarray}}
	\newcommand{\define}{\stackrel{\triangle}{=}}
	\newcommand{\myvec}[1]{\ensuremath{\begin{pmatrix}#1\end{pmatrix}}}
	\newcommand{\mydet}[1]{\ensuremath{\begin{vmatrix}#1\end{vmatrix}}}
	\bibliographystyle{IEEEtran}
	%\bibliographystyle{ieeetr}
	\providecommand{\nCr}[2]{\,^{#1}C_{#2}} % nCr
	\providecommand{\nPr}[2]{\,^{#1}P_{#2}} % nPr
	\providecommand{\mbf}{\mathbf}
	\providecommand{\pr}[1]{\ensuremath{\Pr\left(#1\right)}}
	\providecommand{\qfunc}[1]{\ensuremath{Q\left(#1\right)}}
	\providecommand{\sbrak}[1]{\ensuremath{{}\left[#1\right]}}
	\providecommand{\lsbrak}[1]{\ensuremath{{}\left[#1\right.}}
	\providecommand{\rsbrak}[1]{\ensuremath{{}\left.#1\right]}}
	\providecommand{\brak}[1]{\ensuremath{\left(#1\right)}}
	\providecommand{\lbrak}[1]{\ensuremath{\left(#1\right.}}
	\providecommand{\rbrak}[1]{\ensuremath{\left.#1\right)}}
	\providecommand{\cbrak}[1]{\ensuremath{\left\{#1\right\}}}
	\providecommand{\lcbrak}[1]{\ensuremath{\left\{#1\right.}}
	\providecommand{\rcbrak}[1]{\ensuremath{\left.#1\right\}}}
	\theoremstyle{remark}
	\newtheorem{rem}{Remark}
	\newcommand{\sgn}{\mathop{\mathrm{sgn}}}
	\providecommand{\abs}[1]{\left\vert#1\right\vert}
	\providecommand{\res}[1]{\Res\displaylimits_{#1}} 
	\providecommand{\norm}[1]{\lVert#1\rVert}
	\providecommand{\mtx}[1]{\mathbf{#1}}
	\providecommand{\mean}[1]{E\left[ #1 \right]}
	\providecommand{\fourier}{\overset{\mathcal{F}}{ \rightleftharpoons}}
	\providecommand{\ztrans}{\overset{\mathcal{Z}}{ \rightleftharpoons}}
	%\providecommand{\hilbert}{\overset{\mathcal{H}}{ \rightleftharpoons}}
	\providecommand{\system}{\overset{\mathcal{H}}{ \longleftrightarrow}}
	%\newcommand{\solution}[2]{\textbf{Solution:}{#1}}
	\newcommand{\solution}{\noindent \textbf{Solution: }}
	\providecommand{\dec}[2]{\ensuremath{\overset{#1}{\underset{#2}{\gtrless}}}}
	\numberwithin{equation}{section}
	%\numberwithin{equation}{subsection}
	%\numberwithin{problem}{subsection}
	%\numberwithin{definition}{subsection}
	\makeatletter
	\@addtoreset{figure}{problem}
	\makeatother
	\let\StandardTheFigure\thefigure
	%\renewcommand{\thefigure}{\theproblem.\arabic{figure}}
	\renewcommand{\thefigure}{\theproblem}
	%\numberwithin{figure}{subsection}
	\def\putbox#1#2#3{\makebox[0in][l]{\makebox[#1][l]{}\raisebox{\baselineskip}[0in][0in]{\raisebox{#2}[0in][0in]{#3}}}}
	\def\rightbox#1{\makebox[0in][r]{#1}}
	\def\centbox#1{\makebox[0in]{#1}}
	\def\topbox#1{\raisebox{-\baselineskip}[0in][0in]{#1}}
	\def\midbox#1{\raisebox{-0.5\baselineskip}[0in][0in]{#1}}
	\vspace{3cm}
	\title{ 
		
		OPPENHEIM
	
	}
\author{ DEVANANTH V - EP20BTECH11004$^{}$}	
\maketitle
%\newpage
\tableofcontents

\renewcommand{\thefigure}{\theenumi}
\renewcommand{\thetable}{\theenumi}
%\renewcommand{\theequation}{\thesection}
\bigskip
\begin{abstract}
This is the solution of the question 3.21 (b) and 2.26 (b) of Oppenheim
\end{abstract}
\section{Assignment-1 (3.21 (b))}
\begin{enumerate}
\item Consider a linear time-invarient system with impulse response 
\begin{align}
	h[n]=
	\begin{cases}
		{a^n} & n\geq0 \\
		0 & \ \text{otherwise}
	\end{cases}
\end{align}
and input
\begin{align}
	x[n]=
	\begin{cases}
		1, & 0\leq n \leq\brak{N-1}\\
		0 & \text{otherwise}
	\end{cases}
\end{align}
(b) Determine the output y[n] by computing the inverse z-transform of the product of z-transforms of x[n] and h[n].\\
\solution:
\begin{align}
	H(z)=\sum_{n=-\infty}^{\infty}a^nz^{-n}=\frac{1}{1-az^{-1}}\:\:\abs{z}>\abs{a}
\end{align}
\begin{align}
	X(z)=\sum_{n=-\infty}^{N-1}z^{-n}=\frac{{1-z^{-n}}}{1-z^{-1}}\:\:\abs{x}>0
\end{align}
Therefore,
\begin{align}
	Y(z)=\frac{1-z^{-N}}{\brak{1-az^{-1}}\brak{1-z^{-1}}}=\frac{1}{\brak{1-az^{-1}}\brak{1-z^{-1}}}-\frac{z^{-N}}{\brak{1-az^{-1}}\brak{1-z^{-1}}}\:\:\abs{z}>\abs{a}
\end{align}
Now,
\begin{align}
\frac{1}{\brak{1-az^{-1}}\brak{1-z^{-1}}}=\frac{\frac{1}{1-a^{-1}}}{1-az^{-1}}+\frac{\frac{1}{1-a}}{1-z^{-1}}=\brak{\frac{1}{1-a}}\brak{\frac{1}{1-z^{-1}}-\frac{a}{1-az^{-1}}}	
\end{align}
So,
\begin{align}
	y[n]=\brak{\frac{1}{1-a}}[u[n]-a^{n+1}u[n]-\\u[n-N]-a^{n-N+1}u[n-N]]
	\\=\frac{1-a^{n+1}}{1-a}u[n]-\frac{1-a^{n-N+1}}{1-a}u[n-N]
	\\y[n]=
	\begin{cases}
		0 & n<0\\
		\frac{1-a^{n+1}}{1-a} & 0 \leq n \leq N-1\\
		a^{n+1}\brak{\frac{1-a^{-N}}{a-1}} & n \geq N
	\end{cases}
\end{align}
\end{enumerate}
\section{Assignment-2 (2.26(b))}
\begin{enumerate}
	\item Which of the following discrete-time signals could be eigenfunctions of any stable LTI system?\\
	\begin{enumerate}
		(b)$\:e^{j2\pi n}$
	\end{enumerate}
	\solution:
	\begin{align}
		x[n]=e^{j2\pi\omega n}\\
		y[n]=\sum_{k=-\infty}^{\infty}h[k]e^{j2\omega\brak{n-k}}\\
		=e^{j2\omega n}\sum_{k=-\infty}^{\infty}h[k]e^{-j2\omega k}\\
		=e^{j2\omega n}.H\brak{e^{j2\omega}}
	\end{align}
\end{enumerate}
\end{document}